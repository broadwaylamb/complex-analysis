\documentclass[11pt,openany,a4paper]{scrartcl}

\usepackage{indentfirst}
\usepackage{amsmath,amsthm,amssymb,amsfonts,amsopn}
\usepackage{mathtext}
\usepackage{enumitem}
\usepackage[T1,T2A]{fontenc}
\usepackage[utf8]{inputenc}
\usepackage[english,russian]{babel}
\usepackage[intlimits]{mathtools}
\usepackage[makeroom]{cancel}
\usepackage{titletoc}
\renewcommand{\bfdefault}{sbc}
\usepackage{ccfonts,eulervm,microtype}
\usepackage{tikz}
\usetikzlibrary{arrows}
\usetikzlibrary{calc}

\usepackage[portrait,a4paper,margin=2.5cm,headsep=5mm]{geometry}

\author{Н. В. Цилевич \thanks{Конспект подготовлен студентом Яскевичем С. В.}}
\title{Теория функций комплексной переменной}

\theoremstyle{plain}
\newtheorem{theorem}{Теорема}[section]
\newtheorem{corollary}[theorem]{Следствие}
\newtheorem{proposition}[theorem]{Предложение}
\newtheorem{lemma}[theorem]{Лемма}
\newtheorem{exercise}[theorem]{Упражнение}

\theoremstyle{definition}
\newtheorem{definition}[theorem]{Определение}
\newtheorem{remark}[theorem]{Замечание}
\newtheorem{example}[theorem]{Пример}
\newtheorem{examples}[theorem]{Примеры}
\newtheorem{num}[theorem]{}

\newcommand\mb{\mathbb}
\newcommand\real{\mb R}
\newcommand{\complex}{\mb C}
\newcommand\eqdef{\mathrel{\stackrel{\makebox[0pt]{\mbox{\normalfont\tiny def}}}{=}}}
\newcommand\lparagraph[1]{\paragraph{#1}\mbox{}\\}
\DeclareMathOperator{\Ree}{Re}
\DeclareMathOperator{\Img}{Im}
\DeclareMathOperator{\Arg}{Arg}
\DeclareMathOperator{\dist}{dist}

\begin{document}

\maketitle

\tableofcontents

\pagebreak

\section{Комплексные числа}

Вспомним базовые понятия, связанные с комплексными числами.

Комплексное число представляется в виде пары вещественных чисел: $ z = (x, y) \in \complex$,
где~$ x,y \in \real$. При этом $x$ называется вещественной частью числа $z$, а $y$ — мнимой частью. Комплексные
числа равны тогда и только тогда, когда равны их соответственно вещественные и мнимые части. Также справедливы
следующие соотношения:
\begin{enumerate}
	\item $(x_1, y_1) + (x_2, y_2) = (x_1 + x_2, y_1 + y_2)$
	\item $(x_1, y_1) \cdot (x_2, y_2) = (x_1x_2 - y_1y_2, x_1y_2 + x_2y_1)$
	\item $\real \subset \complex$, при этом $x \in \real \mapsto (x, 0)$ и операции согласованы.
\end{enumerate}

Число $i = (0, 1)$ называется \emph{мнимой единицей}. Легко видеть, что $i^2 = 1$. Таким образом, комплексное
число можно записать в \emph{алгебраической форме}: $z = (x, y) = z + iy$. Числа вида $iy$ $(y \in \real)$
называеются \emph{чисто мнимыми}.

\begin{theorem}
	$(\complex, +, \cdot)$ — поле. Вычитание и деление вводятся как операции, обратные к сложению и умножению.
\end{theorem}
\begin{proof}
	Тривиально.
\end{proof}
\begin{remark}
	На $\complex$ не задано отношения порядка.
\end{remark}

Определим операцию комплексного сопряжения: если $z = x + iy$, то $\overline z = x - iy$.

Свойства сопряжения:
\begin{enumerate}
	\item $\overline{\overline z} = z$ (\emph{инволюция});
	\item $\overline{z_1 \ast z_2} = \overline{z_1} \ast \overline{z_2}$, где $\ast$ — любая арифметическая операция;
	\item $z + \overline z = 2\Ree z$, $z - \overline z = 2i\Img z$;
	\item $z\overline z = (\Ree z)^2 + (\Img z)^2 \geqslant 0$;
	\item $z = \overline z \iff z \in \real$.
\end{enumerate}

\lparagraph{Геометрическая интерпретация комплексных чисел}\mbox{}\\

\begin{center}
	\begin{tikzpicture}[
	    scale=3,
	    axis/.style={very thick, ->, >=stealth'},
	    ]
		\draw[axis] (-0.1, 0)  -- (1.1, 0) node(xline)[right]{$x$};
	    \draw[axis] (0,-0.1) -- (0, 1.1) node(yline)[above]{$y$};
	    \draw[style={->, >=stealth'}] (0, 0) -- (0.8, 0.7) node[right]{$(x, y) \leftarrow x + iy \in \complex$};
	    \draw (0.2, 0) arc (0:41:0.2);
	    \node at (0.35, 0.4) (r) {$r$};
	    \node at (0.27, 0.1) (phi) {$\varphi$};
	\end{tikzpicture}
\end{center}

Сложение комплексных чисел соответствует сложению их радиус-векторов на комплексной плоскости. Перейдём
к полярным координатам: $x = r\cos \varphi$, $y = r \sin \varphi$. Благодаря этому мы можем записать комплексное число в \emph{тригонометрической форме}: $z = r(\cos \varphi + i\sin \varphi)$. $r = |z|$ называется \emph{модулем} числа $z$, а $\varphi$ — его \emph{аргументом}.
\pagebreak

Свойства модуля:
\begin{enumerate}
	\item Геометрический смысл: $|z|$ — расстояние на комплексной плоскости от $0$ до $z$, отсюда $|z_1 - z_2|$ — расстояние между точками $z_1$ и $z_2$;
	\item $|z|=\sqrt{(\Ree z)^2 + (\Img z)^2} = \sqrt{z\overline z} \geqslant 0$;
	\item $|z| = 0 \iff z = 0$;
	\item $|\Ree z|, |\Img z| \leqslant |z|$;
	\item Неравенство треугольника: $|z_1 + z_2| \leqslant |z_1| + |z_2|$;
	\item $|z_1 - z_2| \geqslant ||z_1| - |z_2||$
\end{enumerate}

Заметим, что аргумент определён для любого ненулевого $z$ с точностью до $2\pi k$ ($k \in \mb Z$).
Будем обозначать $\Arg z$ множество всех аргументов $z$, а $\arg z$ — значение аргумента из
фиксированного интервала длины $2\pi$, например, $(0, 2\pi)$.

$$
\Arg z =
\begin{cases}
	\arctg \frac{y}{x} + 2\pi k, k \in \mb Z \text{ — в I и IV квадранта}\\
	\arctg \frac{y}{x} + (2k + 1)\pi, k \in \mb Z \text{ — во II и III квадрантах}\\
\end{cases}
$$
$$
z_1 = z_2 \iff |z_1| = |z_2| \text{ и } \arg z_1 - arg z_2 = 2\pi k\ (k \in \mb Z)
$$

\lparagraph{Показательная форма записи комплексного числа}

Введём теперь обозначение: $e^{i\varphi} \eqdef \cos \varphi + i\sin \varphi$. Тогда $z = re^{i\varphi}$ — такая
форма называется \emph{показательной}.

\begin{lemma}
	Пусть $z_1 = r_1 e^{i\varphi_1}$, $z_2 = r_2 e^{i\varphi_2}$. Тогда $z_1z_2 = r_1r_2e^{i(\varphi_1 +
	\varphi_2)}$, $\frac{z_1}{z_2} = \frac{r_1}{r_2}e^{i(\varphi_1 -
	\varphi_2)}$
\end{lemma}
\begin{proof}
	$$
	z_1z_2 = r_1r_2(\cos \varphi_1 + i\sin \varphi_1)(\cos \varphi_2 + i\sin \varphi_2) =
	$$
	$$
	= r_1r_2(\cos \varphi_1 \cos \varphi_2 - \sin \varphi_1 \sin \varphi_2 + i(\cos \varphi_1 \sin \varphi_2 + \sin \varphi_1 \cos \varphi_2)) =
	$$
	$$
	= r_1r_2(\cos (\varphi_1 + \varphi_2) + i\sin (\varphi_1 + \varphi_2)) = r_1r_2e^{i(\varphi_1 + \varphi_2)}
	$$
	Для частного — аналогично.
\end{proof}

Таким образом, имеем $|z_1z_2| = |z_1||z_2|$, $\Arg (z_1z_2) = \Arg z_1 + \Arg z_2$ (по определению суммы множеств 
$A + B \eqdef \{a + b:\, a \in A,\, b \in B\}$) и $\arg (z_1z_2) = \arg z_1 + \arg z_2 + 2\pi k\:(k\in
\mb Z)$.

Геометрический смысл умножения на число $a \in \complex$: радиус-вектор растягивается в $|a|$ раз и поворачивается на
угол $\arg a$.

\begin{example}
	Умножение на $i$ — поворот на $\frac{\pi}{2}$.
\end{example}

\lparagraph{Корень $n$-й степени из комплексного числа}

Воспользуемся показательной формой: $z^n = r^ne^{in\varphi}$.
\begin{definition}
	$w = \rho e^{i\psi}$ — корень $n$-й степени из $z = re^{i\varphi} \in \complex \textbackslash \{0\}$ тогда и только тогда, когда $w^n = z$.
\end{definition}

То есть:
$$
\rho^ne^{in\psi} = re^{i\varphi} \iff
\begin{cases}
	\rho^n = r\\
	n\psi = \varphi + 2\pi k,\,k \in \mb Z
\end{cases}
\iff
\begin{cases}
	\rho = \sqrt[n]{r}\\
	\psi = \frac{\varphi}{n} + \frac{2\pi k}{n}\quad(k = 0,...,n - 1)
\end{cases}
$$
Итого, корней $n$-й степени из $z$ $n$ штук.

\section{Стереографическая проекция и сфера Римана}

\begin{center}
	\begin{tikzpicture}[
	    scale=4,
	    axis/.style={very thick, ->, >=stealth'},
	    ]
		\draw[axis] (-0.1, 0)  -- (1.1, 0) node(xline)[right]{$x$, $\xi$};
	    \draw[axis] (0,-0.1) -- (0, 1.1) node(zline)[above]{$\zeta$};
	    \draw[axis] (0.05, 0.05) -- (-0.65, -0.65) node(yline)[left]{$y$, $\eta$};
		\draw (0, 0) arc (-90:270:0.4);
		\draw (0.4, 0.4) arc (-48:-131.5:0.6);
		\draw[dotted] (0.4, 0.4) arc (48:131.5:0.6);
		\node at (0, 0.8) (N) {};
		\node at (0.3, 0.85) {$N = (0, 0, 1)$};
		\node at (0.5, -0.4) (z1) {};
		\node at (0.2, -0.7) (z2) {};
		\node at (0.12, 0.53) (Az1) {};
		\node at (0.03, 0.6) (Az2) {};
		\node at ($(z1)+(0.1, 0)$) {$z_1$};
		\node at ($(z2)+(0.1, 0)$) {$z_2$};
		\node at ($(Az1)+(0.15, 0)$) {$A(z_1)$};
		\node at ($(Az2)+(-0.15, 0)$) {$A(z_2)$};
		\draw[dotted] (N) -- (Az1);
		\draw (Az1) -- (z1);
		\draw[dotted] (N) -- (Az2);
		\draw (Az2) -- (z2);
		\fill[black] (N) circle (.5pt);
		\fill[black] (z1) circle (.5pt);
		\fill[black] (z2) circle (.5pt);
		\fill[black] (Az1) circle (.5pt);
		\fill[black] (Az2) circle (.5pt);
	\end{tikzpicture}
\end{center}

Рассмотрим сферу $S$ с центром в точке $(0, 0, \frac{1}{2})$ и радиусом $\frac{1}{2}$. Уравнение этой сферы будет 
таким:
$$
\xi^2 + \eta^2 + \zeta^2 - \zeta = 0
$$

Отождествим комплексную плоскость $(x, y)$ с плоскостью $(\xi, \eta)$. Рассмотрим лучи, исходящие из полюса $N$ в
точки $z$. Ясно, что точка пересечения луча и сферы единственна. Обозначим её как $A(z)$. Это и будет 
стереографическая проекция точки $z$ на сферу $S$, которая называется \emph{сферой Римана}. Таким образом мы 
установили взаимно-однозначное соответствие между комплексной плоскостью и сферой Римана без полюса:
$$
\complex \leftrightarrow S \textbackslash \{N\}
$$

\begin{proposition}
	Справедливы соотношения:
	$$
	\xi = \frac{x}{1 + |z|^2},\quad\eta = \frac{y}{1 + |z|^2},\quad\zeta = \frac{|z^2|}{1 + |z|^2}
	$$
	$$
	x = \frac{\xi}{1 - \zeta},\quad y = \frac{\eta}{1 - \zeta}
	$$
\end{proposition}
\begin{proof}
	Построим прямую через точки $N = (0, 0, 1)$ и $z = (x, y, 0)$. Она имеет вид
	$\{(tx, ty, 1-t)\,|\, t \in \real\}$. Подставим координаты точек прямой в уравнение сферы Римана:
	$$
	t^2x^2 + t^2y^2 + \cancel{1} - 2t + t^2 - \cancel{1} + t = 0
	$$
	$$
	t^2(\underbrace{x^2 + y^2}_{|z|^2} + 1) = t \implies t = \frac{1}{1 + |z|^2} \text{, откуда } \xi = tx =
	\frac{x}{1 + |z|^2}
	$$
	Далее аналогично.
\end{proof}
\begin{proposition}
	$\dist(A(z_1), A(z_2)) = \frac{|z_1 - z_2|}{\sqrt{1 + |z_1|^2} \cdot \sqrt{1 + |z_2|^2}}$,
	$\dist(A(z), N) = \frac{1}{\sqrt{q + |z|^2}}$.
\end{proposition}
\begin{proof}
	Спроецируем $A(z_1)$, $A(z_2)$ на ось $O\zeta$. Видно, что треугольники $\triangle BNA(z_1)$ и $\triangle ONz_1$
	подобны (здесь $B = (0, 0, \zeta_1)$). Поэтому
	$$
	\frac{\dist(N, A(z_1))}{\dist(N, z_1)} = \frac{\overbrace{\dist(N, B)}^{1 - \zeta_1}}{\underbrace{\dist(N, O)}_1}
	\implies \dist(N, A(z_1)) = \sqrt{1 + |z_1|^2} \cdot \overbrace{(1 - \zeta_1)}^{\frac{1}{1 + |z_1|^2}} = 
	\frac{1}{\sqrt{1 + |z_1|^2}}
	$$
	Точно также подобны $\triangle NA(z_1)A(z_2)$ и $\triangle Nz_1z_2$, отсюда
	$$
	\frac{\dist(A(z_1), A(z_2))}{\dist(z_1, z_2)} = \frac{\dist(N, A(z_1))}{\dist(N, z_1)} = \frac{1}{\sqrt{1 +|z_1|^2}}
	\cdot \frac{1}{\sqrt{1 + |z_2|^2}}
	$$
\end{proof}

\begin{definition}
	Обобщённая окружность — это окружность или прямая.
\end{definition}

Запишем уравнение обобщённой окружности:
$$
A(x^2 + y^2) + Bx + Cy + D = 0 \text{, где } A, B, C, D \in \real,\, B^2 + C^2 > 4AD
$$
Очевидно, что это уравнение является уравнением окружности тогда и только тогда, когда $A \neq 0$. В противном случае это — 
прямая.

\begin{proposition}
	Стереографическая проекция устанавливает биекцию между обобщёнными окружностями в $\complex$ и окружностями на сфере 
	Римана. При этом прямым соответствуют окружности, проходящие через точку $N$.
\end{proposition}
\begin{proof}
	Воспользуемся формулами $x = \frac{\xi}{1 - \zeta}$, $y = \frac{\eta}{1 - \zeta}$:
	$$
	\frac{A\cdot(\xi^2 + \eta^2)}{(1 - \zeta)^2} + \frac{B\xi + C\eta}{1 - \zeta} + D = 0
	$$
	С учётом $\xi^2 + \eta^2 = \zeta(1 - \zeta)$ получим 
	$$
	\frac{A\zeta}{1 - \zeta} + \frac{B\xi + C\eta}{1 - \zeta} + D = 0
	$$
	$$
	(A - D)\zeta + B\xi + C\eta + D = 0
	$$
	Мы получили уравнение плоскости. Значит, образом будет пересечение сферы с плоскостью, то есть окружность.
	Если мы в полученное уравнение подставим $N$, то убедимся, что $N$ лежит в этой плоскости, при этом $A = 0$.
\end{proof}

\begin{definition}
	$\overline{\complex} = \complex \cup \{\infty\}$ называется расширенной комплексной плоскостью, $\infty$ — бесконечно 
	удалённая точка.
\end{definition}

Дополним определение стереографической проекции: пусть $N$ переходит в $\infty$ и обратно. Тогда стереографическая проекция 
устанавливает биекцию между $S$ и $\overline{\complex}$, следовательно мы можем считать прямую окружностью, проходящей через
$\infty$.

\begin{remark}
	Стереографическая проекция конформна, то есть сохраняет углы между кривыми (будет описано далее).
\end{remark}
\begin{remark}
	Дробно-линейные отображения в $\complex$ переходят в движения сферы Римана.
\end{remark}
\pagebreak

\section{Предел и непрерывность}

Мы отождествили $\complex$ и $\real^2$, а следовательно ввели понятие сходимости, которое наследует из $\real^2$ основные 
свойства.
\begin{definition}
	$z_n \rightarrow z \iff \forall\varepsilon > 0\quad \exists N: \forall n > N\quad |z_n - z| < \varepsilon$.
\end{definition}

Свойства сходимости:
\begin{enumerate}
	\item \emph{Покоординатность}: $z_n \rightarrow z \iff x_n \rightarrow x,\,y_n \rightarrow y$ (где $z_n = x_n + iy_n$,
	$z = x + iy$);
	\item \emph{Критерий Коши}: $z_n$ сходится, если $\forall \varepsilon > 0\quad\exists N: \forall m, n > N\quad
	|z_n - z_m| < \varepsilon$;
	\item \emph{Принцип Больцано-Вейерштрасса}: множество $A$ ограничено тогда и только тогда, когда существует $R$ такое, 
	что $z < R$ для всех $z \in A$. Если $z_n$ ограничена, то из $z_n$ можно выбрать сходящуюся подпоследовательность;
	\item $\lim (z_k \ast z'_k) = \lim z_k \ast \lim z'_k$, где $\ast$ — арифметическая операция.
\end{enumerate}

Расширим понятие сходимости на $\overline{\complex}$.

\begin{definition}
	$z_n \in \complex \rightarrow \infty \iff \forall R > 0\quad\exists N: \forall n > N\quad |z_n| > R$.
\end{definition}
\begin{remark}
	Очевидно, что $z_n \rightarrow \infty \iff |z_n| \rightarrow \infty \iff \frac{1}{|z_n|} \rightarrow 0 \iff
	\frac{1}{z_n} \rightarrow 0$.
\end{remark}

\begin{proposition}
	Сходимость в $\overline{\complex}$ равносильна сходимости на сфере Римана. В частности, $z_n \rightarrow \infty \iff
	A(z_n) \rightarrow N$.
\end{proposition}
\begin{proof}
	Следует из формул для расстояния:
	$$
	\dist(A(z_n), A(z)) = \frac{|z_n - z|}{\sqrt{1 + |z_n|^2} \cdot \sqrt{1 + |z|^2}} \rightarrow 0
	$$
\end{proof}

Отсюда также вытекает, что $\overline{\complex}$ — компактно.

Займёмся теперь изучением функций комплексной переменной.

\lparagraph{Предел функции комплексной переменной}
\begin{definition}
	$f(x + iy) = \underbrace{u(x, y)}_{\Ree f} + i\underbrace{v(x, y)}_{\Img f}$, при этом $u, v: \real^2 \rightarrow \real$
\end{definition}
\begin{definition}
	Пусть $E\subset\complex$, $f: E \rightarrow \complex$, $z_0$ — предельная точка $E$ и $a \in \overline\complex$.
	
	$\lim\limits_{z \rightarrow z_0} f(z) = a \iff \forall \varepsilon > 0\quad \exists \delta > 0: z \in \overset{\circ}{B}_\delta(z_0)\cap E \implies f(z) \in B_\varepsilon(a)$
\end{definition}

Имеет место и покоординатная сходимость: $$
\lim_{z \rightarrow z_0} f(z) = a = \alpha + i\beta \iff 
\begin{cases}
	\lim\limits_{(x, y) \rightarrow (x_0, y_0)} u(x, y) = \alpha\\
	\lim\limits_{(x, y) \rightarrow (x_0, y_0)} v(x, y) = \beta\\
\end{cases}
$$

\begin{definition}[по Гейне]
	$\lim\limits_{z \rightarrow z_0} f(z) = a$, если для любой последовательности $z_n \subset E$ такой, что $z_n 
	\rightarrow z_0$ выполнено $f(z_n) \rightarrow a$.
\end{definition}
\begin{definition}
	Пусть $E \subset \complex$, $f: E \rightarrow \complex$, $z_0$ — неизолированная точка множества $E$. Функция $f$ 
	называется \emph{непрерывной в точке} $z_0$, если $\lim\limits_{z \rightarrow z_0} f(z) = f(z_0)$; \emph{непрерывной на 
	множестве} $E$, если $f$ непрерывна в каждой точке этого множества.
\end{definition}

Отметим важные свойства непрерывности:
\begin{enumerate}
	\item Функция $f$ непрерывна тогда и только тогда, когда $\Ree f$ и $\Img f$ непрерывны по совокупности переменных;
	\item Композиция непрерывных функций непрерывна;
	\item Если функция непрерывна на компакте, то она на нём ограничена, а её модуль достигает на этом компакте своих 
	наибольшего и наименьшего значений.
	\item Если $G$ — область (т. е. открытое связное множество), $f: G \rightarrow D$ и $f$ — непрерывная биекция, то
	$D$ — тоже область и $f^{-1}$ непрерывно.
\end{enumerate}

\section{Дифференцирование функции комплексной переменной}

\end{document}
