\documentclass[11pt,openany,a4paper]{scrartcl}

\usepackage{indentfirst}
\usepackage{amsmath,amsthm,amssymb,amsfonts,amsopn}
\usepackage{mathtext}
\usepackage{enumitem}
\usepackage[T1,T2A]{fontenc}
\usepackage[utf8]{inputenc}
\usepackage[english,russian]{babel}
\usepackage[intlimits]{mathtools}
\usepackage[makeroom]{cancel}
\usepackage{titletoc}
\renewcommand{\bfdefault}{sbc}
\usepackage{ccfonts,eulervm,microtype}
\usepackage{enumitem}
\usepackage{tikz}
\usetikzlibrary{arrows}
\usetikzlibrary{calc}

\tikzset{
    pil/.style={
           ->,
           thick,
           shorten <=2pt,
           shorten >=2pt},
    axis/.style={very thick, ->, >=stealth'},
}

\usepackage[portrait,a4paper,margin=2.5cm,headsep=5mm]{geometry}

\author{Н. В. Цилевич \thanks{Конспект подготовлен студентом Яскевичем С. В.}}
\title{Теория функций комплексной переменной}

\theoremstyle{plain}
\newtheorem{theorem}{Теорема}[section]
\newtheorem{corollary}[theorem]{Следствие}
\newtheorem{proposition}[theorem]{Предложение}
\newtheorem{lemma}[theorem]{Лемма}
\newtheorem{exercise}[theorem]{Упражнение}

\theoremstyle{definition}
\newtheorem{definition}[theorem]{Определение}
\newtheorem{remark}[theorem]{Замечание}
\newtheorem{example}[theorem]{Пример}
\newtheorem{examples}[theorem]{Примеры}
\newtheorem{num}[theorem]{}

\newcommand\mb{\mathbb}
\newcommand\real{\mb R}
\newcommand{\complex}{\mb C}
\newcommand\eqdef{\mathrel{\stackrel{\makebox[0pt]{\mbox{\normalfont\tiny def}}}{=}}}
\newcommand\lparagraph[1]{\paragraph{#1}\mbox{}\\}
\newcommand{\pd}[2]{\frac{\partial #1}{\partial #2}}
\DeclareMathOperator{\Ree}{Re}
\DeclareMathOperator{\Img}{Im}
\DeclareMathOperator{\Arg}{Arg}
\DeclareMathOperator{\dist}{dist}
\DeclareMathOperator{\const}{const}
\DeclareMathOperator{\Ln}{Ln}

\begin{document}

\maketitle

\tableofcontents

\pagebreak

\section{Комплексные числа}

Вспомним базовые понятия, связанные с комплексными числами.

Комплексное число представляется в виде пары вещественных чисел: $ z = (x, y) \in \complex$,
где~$ x,y \in \real$. При этом $x$ называется вещественной частью числа $z$, а $y$ — мнимой частью. Комплексные
числа равны тогда и только тогда, когда равны их соответственно вещественные и мнимые части. Также справедливы
следующие соотношения:
\begin{enumerate}
	\item $(x_1, y_1) + (x_2, y_2) = (x_1 + x_2, y_1 + y_2)$
	\item $(x_1, y_1) \cdot (x_2, y_2) = (x_1x_2 - y_1y_2, x_1y_2 + x_2y_1)$
	\item $\real \subset \complex$, при этом $x \in \real \mapsto (x, 0)$ и операции согласованы.
\end{enumerate}

Число $i = (0, 1)$ называется \emph{мнимой единицей}. Легко видеть, что $i^2 = -1$. Таким образом, комплексное
число можно записать в \emph{алгебраической форме}: $z = (x, y) = z + iy$. Числа вида $iy$ $(y \in \real)$
называеются \emph{чисто мнимыми}.

\begin{theorem}
	$(\complex, +, \cdot)$ — поле. Вычитание и деление вводятся как операции, обратные к сложению и умножению.
\end{theorem}
\begin{proof}
	Тривиально.
\end{proof}
\begin{remark}
	На $\complex$ не задано отношения порядка.
\end{remark}

Определим операцию комплексного сопряжения: если $z = x + iy$, то $\overline z = x - iy$.

Свойства сопряжения:
\begin{enumerate}
	\item $\overline{\overline z} = z$ (\emph{инволюция});
	\item $\overline{z_1 \ast z_2} = \overline{z_1} \ast \overline{z_2}$, где $\ast$ — любая арифметическая операция;
	\item $z + \overline z = 2\Ree z$, $z - \overline z = 2i\Img z$;
	\item $z\overline z = (\Ree z)^2 + (\Img z)^2 \geqslant 0$;
	\item $z = \overline z \iff z \in \real$.
\end{enumerate}

\lparagraph{Геометрическая интерпретация комплексных чисел}\mbox{}\\

\begin{center}
	\begin{tikzpicture}[scale=3]
		\draw[axis] (-0.1, 0)  -- (1.1, 0) node(xline)[right]{$x$};
	    \draw[axis] (0,-0.1) -- (0, 1.1) node(yline)[above]{$y$};
	    \draw[style={->, >=stealth'}] (0, 0) -- (0.8, 0.7) node[right]{$(x, y) \leftarrow x + iy \in \complex$};
	    \draw (0.2, 0) arc (0:41:0.2);
	    \node at (0.35, 0.4) (r) {$r$};
	    \node at (0.27, 0.1) (phi) {$\varphi$};
	\end{tikzpicture}
\end{center}

Сложение комплексных чисел соответствует сложению их радиус-векторов на комплексной плоскости. Перейдём
к полярным координатам: $x = r\cos \varphi$, $y = r \sin \varphi$. Благодаря этому мы можем записать комплексное
число в \emph{тригонометрической форме}: $z = r(\cos \varphi + i\sin \varphi)$. $r = |z|$ называется
\emph{модулем} числа $z$, а $\varphi$ — его \emph{аргументом}.
\pagebreak

Свойства модуля:
\begin{enumerate}
	\item Геометрический смысл: $|z|$ — расстояние на комплексной плоскости от $0$ до $z$, отсюда
	$|z_1 - z_2|$ — расстояние между точками $z_1$ и $z_2$;
	\item $|z|=\sqrt{(\Ree z)^2 + (\Img z)^2} = \sqrt{z\overline z} \geqslant 0$;
	\item $|z| = 0 \iff z = 0$;
	\item $|\Ree z|, |\Img z| \leqslant |z|$;
	\item Неравенство треугольника: $|z_1 + z_2| \leqslant |z_1| + |z_2|$;
	\item $|z_1 - z_2| \geqslant ||z_1| - |z_2||$
\end{enumerate}

Заметим, что аргумент определён для любого ненулевого $z$ с точностью до $2\pi k$ ($k \in \mb Z$).
Будем обозначать $\Arg z$ множество всех аргументов $z$, а $\arg z$ — значение аргумента из
фиксированного интервала длины $2\pi$, например, $[0, 2\pi)$.

$$
\Arg z =
\begin{cases}
	\arctg \frac{y}{x} + 2\pi k, k \in \mb Z \text{ — в I и IV квадрантах}\\
	\arctg \frac{y}{x} + (2k + 1)\pi, k \in \mb Z \text{ — во II и III квадрантах}\\
\end{cases}
$$
$$
z_1 = z_2 \iff |z_1| = |z_2| \text{ и } \arg z_1 - \arg z_2 = 2\pi k\ (k \in \mb Z)
$$

\lparagraph{Показательная форма записи комплексного числа}

Введём теперь обозначение: $e^{i\varphi} \eqdef \cos \varphi + i\sin \varphi$. Тогда $z = re^{i\varphi}$ — такая
форма называется \emph{показательной}.

\begin{lemma}
	Пусть $z_1 = r_1 e^{i\varphi_1}$, $z_2 = r_2 e^{i\varphi_2}$. Тогда $z_1z_2 = r_1r_2e^{i(\varphi_1 +
	\varphi_2)}$, $\frac{z_1}{z_2} = \frac{r_1}{r_2}e^{i(\varphi_1 -
	\varphi_2)}$.
\end{lemma}
\begin{proof}
	$$
	z_1z_2 = r_1r_2(\cos \varphi_1 + i\sin \varphi_1)(\cos \varphi_2 + i\sin \varphi_2) =
	$$
	$$
	= r_1r_2(\cos \varphi_1 \cos \varphi_2 - \sin \varphi_1 \sin \varphi_2 + i(\cos \varphi_1 \sin \varphi_2 +
	\sin \varphi_1 \cos \varphi_2)) =
	$$
	$$
	= r_1r_2(\cos (\varphi_1 + \varphi_2) + i\sin (\varphi_1 + \varphi_2)) = r_1r_2e^{i(\varphi_1 + \varphi_2)}
	$$
	Для частного — аналогично.
\end{proof}

Таким образом, имеем $|z_1z_2| = |z_1||z_2|$, $\Arg (z_1z_2) = \Arg z_1 + \Arg z_2$ (по определению суммы множеств
$A + B \eqdef \{a + b\,|\, a \in A,\, b \in B\}$) и $\arg (z_1z_2) = \arg z_1 + \arg z_2 + 2\pi k\:(k\in
\mb Z)$.

Геометрический смысл умножения на число $a \in \complex$: радиус-вектор растягивается в $|a|$ раз и поворачивается на
угол $\arg a$.

\begin{example}
	Умножение на $i$ — поворот на $\frac{\pi}{2}$.
\end{example}

\lparagraph{Корень $n$-й степени из комплексного числа}

Воспользуемся показательной формой: $z^n = r^ne^{in\varphi}$.
\begin{definition}
	$w = \rho e^{i\psi}$ — корень $n$-й степени из $z = re^{i\varphi} \in \complex \textbackslash \{0\}$ тогда и
	только тогда, когда $w^n = z$.
\end{definition}

То есть:
$$
\rho^ne^{in\psi} = re^{i\varphi} \iff
\begin{cases}
	\rho^n = r\\
	n\psi = \varphi + 2\pi k,\,k \in \mb Z
\end{cases}
\iff
\begin{cases}
	\rho = \sqrt[n]{r}\\
	\psi = \frac{\varphi}{n} + \frac{2\pi k}{n}\quad(k = 0,...,n - 1)
\end{cases}
$$
Итого, корней $n$-й степени из $z$ $n$ штук.

\section{Стереографическая проекция и сфера Римана}

\begin{center}
	\begin{tikzpicture}[scale=4]
		\draw[axis] (-0.1, 0)  -- (1.1, 0) node(xline)[right]{$x$, $\xi$};
	    \draw[axis] (0,-0.1) -- (0, 1.1) node(zline)[above]{$\zeta$};
	    \draw[axis] (0.05, 0.05) -- (-0.65, -0.65) node(yline)[left]{$y$, $\eta$};
		\draw (0, 0) arc (-90:270:0.4);
		\draw (0.4, 0.4) arc (-48:-131.5:0.6);
		\draw[dotted] (0.4, 0.4) arc (48:131.5:0.6);
		\node at (0, 0.8) (N) {};
		\node at (0.3, 0.85) {$N = (0, 0, 1)$};
		\node at (0.5, -0.4) (z1) {};
		\node at (0.2, -0.7) (z2) {};
		\node at (0.12, 0.53) (Az1) {};
		\node at (0.03, 0.6) (Az2) {};
		\node at ($(z1)+(0.1, 0)$) {$z_1$};
		\node at ($(z2)+(0.1, 0)$) {$z_2$};
		\node at ($(Az1)+(0.15, 0)$) {$A(z_1)$};
		\node at ($(Az2)+(-0.15, 0)$) {$A(z_2)$};
		\draw[dotted] (N) -- (Az1);
		\draw (Az1) -- (z1);
		\draw[dotted] (N) -- (Az2);
		\draw (Az2) -- (z2);
		\fill[black] (N) circle (.5pt);
		\fill[black] (z1) circle (.5pt);
		\fill[black] (z2) circle (.5pt);
		\fill[black] (Az1) circle (.5pt);
		\fill[black] (Az2) circle (.5pt);
	\end{tikzpicture}
\end{center}

Рассмотрим сферу $S$ с центром в точке $(0, 0, \frac{1}{2})$ и радиусом $\frac{1}{2}$. Уравнение этой сферы будет
таким:
$$
\xi^2 + \eta^2 + \zeta^2 - \zeta = 0
$$

Отождествим комплексную плоскость $(x, y)$ с плоскостью $(\xi, \eta)$. Рассмотрим лучи, исходящие из полюса $N$ в
точки $z \in \complex$. Ясно, что точка пересечения луча и сферы единственна. Обозначим её как $A(z)$. Это и будет
стереографическая проекция точки $z$ на сферу $S$, которая называется \emph{сферой Римана}. Таким образом мы
установили взаимно-однозначное соответствие между комплексной плоскостью и сферой Римана без полюса:
$$
\complex \leftrightarrow S \textbackslash \{N\}
$$

\begin{proposition}
	Справедливы соотношения:
	$$
	\xi = \frac{x}{1 + |z|^2},\quad\eta = \frac{y}{1 + |z|^2},\quad\zeta = \frac{|z^2|}{1 + |z|^2}
	$$
	$$
	x = \frac{\xi}{1 - \zeta},\quad y = \frac{\eta}{1 - \zeta}
	$$
\end{proposition}
\begin{proof}
	Построим прямую через точки $N = (0, 0, 1)$ и $z = (x, y, 0)$. Она имеет вид
	$\{(tx, ty, 1-t)\,|\, t \in \real\}$. Подставим координаты точек прямой в уравнение сферы Римана:
	$$
	t^2x^2 + t^2y^2 + \cancel{1} - 2t + t^2 - \cancel{1} + t = 0
	$$
	$$
	t^2(\underbrace{x^2 + y^2}_{|z|^2} + 1) = t \implies t = \frac{1}{1 + |z|^2} \text{, откуда } \xi = tx =
	\frac{x}{1 + |z|^2}
	$$
	Далее аналогично.
\end{proof}
\begin{proposition}
	$\dist(A(z_1), A(z_2)) = \frac{|z_1 - z_2|}{\sqrt{1 + |z_1|^2} \cdot \sqrt{1 + |z_2|^2}}$,
	$\dist(A(z), N) = \frac{1}{\sqrt{1 + |z|^2}}$.
\end{proposition}
\begin{proof}
	Спроецируем $A(z_1)$, $A(z_2)$ на ось $O\zeta$. Видно, что треугольники $\triangle BNA(z_1)$ и $\triangle ONz_1$
	подобны (здесь $B = (0, 0, \zeta_1)$). Поэтому
	$$
	\frac{\dist(N, A(z_1))}{\dist(N, z_1)} = \frac{\overbrace{\dist(N, B)}^{1 - \zeta_1}}{\underbrace{\dist(N, O)}_1}
	\implies \dist(N, A(z_1)) = \sqrt{1 + |z_1|^2} \cdot \overbrace{(1 - \zeta_1)}^{\frac{1}{1 + |z_1|^2}} =
	\frac{1}{\sqrt{1 + |z_1|^2}}
	$$
	Точно также подобны $\triangle NA(z_1)A(z_2)$ и $\triangle Nz_1z_2$, отсюда
	$$
	\frac{\dist(A(z_1), A(z_2))}{\dist(z_1, z_2)} = \frac{\dist(N, A(z_1))}{\dist(N, z_1)} = \frac{1}{\sqrt{1 +|z_1|^2}}
	\cdot \frac{1}{\sqrt{1 + |z_2|^2}}
	$$
\end{proof}

\begin{definition}
	Обобщённая окружность — это окружность или прямая.
\end{definition}

Запишем уравнение обобщённой окружности:
$$
A(x^2 + y^2) + Bx + Cy + D = 0 \text{, где } A, B, C, D \in \real,\, B^2 + C^2 > 4AD
$$
Очевидно, что это уравнение является уравнением окружности тогда и только тогда, когда $A \neq 0$. В противном случае это —
прямая.

\begin{proposition}
	Стереографическая проекция устанавливает биекцию между обобщёнными окружностями в $\complex$ и окружностями на сфере
	Римана. При этом прямым соответствуют окружности, проходящие через точку $N$.
\end{proposition}
\begin{proof}
	Воспользуемся формулами $x = \frac{\xi}{1 - \zeta}$, $y = \frac{\eta}{1 - \zeta}$:
	$$
	\frac{A\cdot(\xi^2 + \eta^2)}{(1 - \zeta)^2} + \frac{B\xi + C\eta}{1 - \zeta} + D = 0
	$$
	С учётом $\xi^2 + \eta^2 = \zeta(1 - \zeta)$ получим
	$$
	\frac{A\zeta}{1 - \zeta} + \frac{B\xi + C\eta}{1 - \zeta} + D = 0
	$$
	$$
	(A - D)\zeta + B\xi + C\eta + D = 0
	$$
	Мы получили уравнение плоскости. Значит, образом будет пересечение сферы с плоскостью, то есть окружность.
	Если мы в полученное уравнение подставим $N$, то убедимся, что $N$ лежит в этой плоскости, при этом $A = 0$.
\end{proof}

\begin{definition}
	$\overline{\complex} = \complex \cup \{\infty\}$ называется расширенной комплексной плоскостью, $\infty$ — бесконечно
	удалённая точка.
\end{definition}

Дополним определение стереографической проекции: пусть $N$ переходит в $\infty$ и обратно. Тогда стереографическая проекция
устанавливает биекцию между $S$ и $\overline{\complex}$, следовательно мы можем считать прямую окружностью, проходящей через
$\infty$.

\begin{remark}
	Стереографическая проекция конформна, то есть сохраняет углы между кривыми (будет описано далее).
\end{remark}
\begin{remark}
	Дробно-линейные отображения в $\complex$ переходят в движения сферы Римана.
\end{remark}
\pagebreak

\section{Предел и непрерывность}

Мы отождествили $\complex$ и $\real^2$, а следовательно ввели понятие сходимости, которое наследует из $\real^2$ основные
свойства.
\begin{definition}
	$z_n \to z \iff \forall\varepsilon > 0\quad \exists N: \forall n > N\quad |z_n - z| < \varepsilon$.
\end{definition}

Свойства сходимости:
\begin{enumerate}
	\item \emph{Покоординатность}: $z_n \to z \iff x_n \to x,\,y_n \to y$ (где $z_n = x_n + iy_n$,
	$z = x + iy$);
	\item \emph{Критерий Коши}: $z_n$ сходится, если $\forall \varepsilon > 0\quad\exists N: \forall m, n > N\quad
	|z_n - z_m| < \varepsilon$;
	\item \emph{Принцип Больцано-Вейерштрасса}: множество $A$ ограничено тогда и только тогда, когда существует $R$ такое,
	что $z < R$ для всех $z \in A$. Если $z_n$ ограничена, то из $z_n$ можно выбрать сходящуюся подпоследовательность;
	\item $\lim (z_k \ast z'_k) = \lim z_k \ast \lim z'_k$, где $\ast$ — арифметическая операция.
\end{enumerate}

Расширим понятие сходимости на $\overline{\complex}$.

\begin{definition}
	$z_n \in \complex \to \infty \iff \forall R > 0\quad\exists N: \forall n > N\quad |z_n| > R$.
\end{definition}
\begin{remark}
	Очевидно, что $z_n \to \infty \iff |z_n| \to \infty \iff \frac{1}{|z_n|} \to 0 \iff
	\frac{1}{z_n} \to 0$.
\end{remark}

\begin{proposition}
	Сходимость в $\overline{\complex}$ равносильна сходимости на сфере Римана. В частности, $z_n \to \infty \iff
	A(z_n) \to N$.
\end{proposition}
\begin{proof}
	Следует из формул для расстояния:
	$$
	\dist(A(z_n), A(z)) = \frac{|z_n - z|}{\sqrt{1 + |z_n|^2} \cdot \sqrt{1 + |z|^2}} \to 0
	$$
\end{proof}

Отсюда также вытекает, что $\overline{\complex}$ — компактно.

Займёмся теперь изучением функций комплексной переменной.

\lparagraph{Предел функции комплексной переменной}
\begin{definition}
	$f(x + iy) = \underbrace{u(x, y)}_{\Ree f} + i\underbrace{v(x, y)}_{\Img f}$, при этом $u, v: \real^2 \to \real$
\end{definition}
\begin{definition}
	Пусть $E\subset\complex$, $f: E \to \complex$, $z_0$ — предельная точка $E$ и $a \in \overline\complex$.

	$\lim\limits_{z \to z_0} f(z) = a \iff \forall \varepsilon > 0\quad \exists \delta > 0: z \in
	\overset{\circ}{B}_\delta(z_0)\cap E \implies f(z) \in B_\varepsilon(a)$
\end{definition}

Имеет место и покоординатная сходимость: $$
\lim_{z \to z_0} f(z) = a = \alpha + i\beta \iff
\begin{cases}
	\lim\limits_{(x, y) \to (x_0, y_0)} u(x, y) = \alpha\\
	\lim\limits_{(x, y) \to (x_0, y_0)} v(x, y) = \beta\\
\end{cases}
$$

\begin{definition}[по Гейне]
	$\lim\limits_{z \to z_0} f(z) = a$, если для любой последовательности $z_n \subset E$ такой, что $z_n
	\to z_0$ выполнено $f(z_n) \to a$.
\end{definition}
\begin{definition}
	Пусть $E \subset \complex$, $f: E \to \complex$, $z_0$ — неизолированная точка множества $E$. Функция $f$
	называется \emph{непрерывной в точке} $z_0$, если $\lim\limits_{z \to z_0} f(z) = f(z_0)$; \emph{непрерывной на
	множестве} $E$, если $f$ непрерывна в каждой точке этого множества.
\end{definition}

Отметим важные свойства непрерывности:
\begin{enumerate}
	\item Функция $f$ непрерывна тогда и только тогда, когда $\Ree f$ и $\Img f$ непрерывны по совокупности переменных;
	\item Композиция непрерывных функций непрерывна;
	\item Если функция непрерывна на компакте, то она на нём ограничена, а её модуль достигает на этом компакте своих
	наибольшего и наименьшего значений.
	\item Если $G$ — область (т. е. открытое связное множество), $f: G \to D$ и $f$ — непрерывная биекция, то
	$D$ — тоже область и $f^{-1}$ непрерывно.
\end{enumerate}

\section{Дифференцирование функции комплексной переменной}

\begin{definition}
	Пусть $f: G \to \complex$, $G$ — область, $z_0 \in G$. Функция называется \emph{дифференцируемой в точке $z_0$},
	если существует предел $\lim\limits_{z \to z_0} \frac{f(z) - f(z_0)}{z - z_0}$, который называется
	\emph{производной} функции $f$ в точке $z_0$ и обозначается $f'(z_0)$.
\end{definition}

Введём обозначение: $z - z_0 = \Delta z$, $f(z) - f(z_0) = \Delta f$ и заметим, что если $\varphi = \alpha +
i\beta = o(\Delta z)$, то это то же самое, что $\alpha, \beta = o(\sqrt{\Delta x^2 + \Delta y^2})$.

Рассмотрим простейшие свойства дифференцируемых функций:
\begin{enumerate}
	\item \emph{Определение через дифференциал}: функция $f$ дифференцируема в точке $z_0$ тогда и только тогда,
	когда существет точка $A \in \complex$ такая, что $\Delta f = A\cdot \Delta z + o(\Delta z)$. При этом
	$A = f'(z_0)$;
	\item Если $f$ дифференцируема в точке, то она непрерывна в этой точке;
	\item Сумма и произведение дифференцируемых функций дифференцируемы;
	\item \emph{Композиция}: пусть $f: G \to D$, $g: D \to \complex$, $G$ и $D$ — области, $z_0 \in G$, $w_0 = f(z_0)$ и
	$h(z) = g(f(z))$. Если $f$ дифференцируема в точке $z_0$, $g$ дифференцируема в точке $w_0$, то $h$ дифференцируема
	в точке $z_0$ и $h'(z_0) = g'(w_0) \cdot f'(z_0)$.
\end{enumerate}

\begin{theorem}
	Пусть $f = u + iv: G  \to \complex$, $G$ — область, $z_0 = (x_0, y_0) \in G$. $f$ дифференцируема в точке
	$z_0$ тогда и только тогда, когда $u$ и $v$ дифференцируемы как функции из $\real^2$ в $\real$ и, кроме того,
	выполнены \textbf{условия Коши-Римана}:
	$$
	\frac{\partial u}{\partial x} (x_0, y_0) = \frac{\partial v}{\partial y} (x_0, y_0),\quad
	\frac{\partial u}{\partial y}(x_0, y_0) = -\frac{\partial v}{\partial x} (x_0, y_0)
	$$
\end{theorem}
\begin{proof}
	Дифференцируемость $f$ в точке $z_0$ равносильна существованию\\$A = a + ib \in \complex$ такого, что
	$\Delta f = A\Delta z + \varphi$, где $\varphi = o(\Delta z)$. Пусть $\varphi = \alpha + i\beta$. Тогда это
	будет равносильно $\Delta u + i\Delta v = (a + ib)(\Delta x + i\Delta y) + \alpha + i\beta$, причём $\alpha,
	\beta = o(\sqrt{\Delta x^2 + \Delta y^2})$, что, в свою очередь, равносильно выполнению условий
	$$
	\begin{cases}
		\Delta u = a\Delta x - b\Delta y + \alpha\\
		\Delta v = a\Delta y + b\Delta x + \beta\\
	\end{cases}
	$$
	То есть $u$ и $v$ дифференцируемы в точке $(x_0, y_0)$. Легко видеть, что
	$$
	\frac{\partial u}{\partial x} (x_0, y_0) = a = \frac{\partial v}{\partial y} (x_0, y_0), \quad
	-\frac{\partial u}{\partial y} (x_0, y_0) = b = \frac{\partial v}{\partial x}(x_0, y_0).
	$$
\end{proof}
\begin{remark}
	Производную функции $f$ можно теперь выразить так: $$
	f'(z_0) = \pd{u}{x}(x_0, y_0) + i\pd{v}{x}(x_0, y_0) = \pd{v}{y}(x_0, y_0) + i\pd{v}{x}(x_0, y_0) =
	$$
	$$
	= \pd{u}{x}(x_0, y_0) - i\pd{u}{y}(x_0, y_0) = \pd{v}{y}(x_0, y_0) - i\pd{u}{y}(x_0, y_0)
	$$
\end{remark}
\begin{definition}
	Функция называется \emph{голоморфной в точке}, если она дифференцируема в некоторой окрестности этой точки.
	Функция называется \emph{голоморфной в области}, если она дифференцируема в любой точке этой области.
\end{definition}
Далее будем обозначать множество всех функций, голоморфных в области $G$, как $H(G)$.

\begin{example}
	$$
	f(z) = u + iv = \overline z = x - iy
	$$
	$$
	u(x, y) = x,\quad v(x, y) = -y,\quad u'_x = 1, v'_y = -1
	$$
	Видно, что условия Коши-Римана не выполнены — функция нигде не дифференцируема.
\end{example}
Попробуйте в качестве упражнения доказать, что $f(z) = |z|^2$ дифференцируема, но не голоморфна в точке $0$.

\begin{proposition}[Условия Коши-Римана в тригонометрической форме]
\mbox{}\\
	Пусть $f(z) = u(r, \varphi) + iv(r, \varphi)$. Тогда $\pd{u}{\varphi} = -r\pd{v}{r}$,
	$\pd{v}{\varphi} = r\pd{u}{r}$
\end{proposition}
\begin{proof}
	$x = r\cos \varphi$, $y = r\sin \varphi$.
	$$
	u'_\varphi = u'_x\cdot x'_\varphi + u'_y\cdot y'_\varphi = -u'_x \cdot r\sin \varphi +
	u'_y \cdot r\cos \varphi = r\cdot(-v'_y\sin \varphi - v'_x\cos \varphi)
	$$
	С учётом $v'_r = v'_x \cdot x'_r + v'_y \cdot y'_r = v'_x\cdot \cos \varphi + v'_y \sin \varphi$
	получим:
	$$
	r\cdot(-v'_y\sin \varphi - v'_x\cos \varphi) = -rv'_r
	$$
	Аналогично и для $\pd{v}{\varphi} = r\pd{u}{r}$.
\end{proof}

\begin{example}
	$$
	f(z) = z^n = r^ne^{i\varphi n} = r^n(\cos n\varphi + i\sin n\varphi)
	$$
	$$
	u = r^n\cos n\varphi,\quad v = r^n\sin n\varphi
	$$
	$$
	u'_\varphi = -nr^n\sin n\varphi,\quad v'_r = nr^{n-1}\sin n\varphi,\quad u'_\varphi = -rv'_r.
	$$
\end{example}
\begin{definition}
	Функция $f$ называется \emph{регулярной в точке}, если она голоморфна в этой точке и $f'$ непрерывна в некоторой
	окрестности этой точки.
\end{definition}
\begin{theorem}[об обратной функции]
	Пусть функция $f$ регулярна в точке $z_0$ и $f'(z_0) \neq 0$. Тогда существует окрестность точки $z_0$, в которой
	$f$ обратима, причём $f^{-1}$ в соответствующей окрестности точки $w_0 = f(z_0)$ дифференцируема и
	$(f^{-1})'(w_0) = \frac{1}{f'(z_0)}$.
\end{theorem}
\begin{proof}
	Мы хотим применить вещественную теорему об обратной функции, рассматривая $f$ как $\real^2 \to \real^2$. Для этого
	нам нужна гладкость, которая есть по условию теоремы, и ненулевой якобиан.
	$$
	\begin{vmatrix}
		u'_x & u'_y\\
		v'_x & v'_y
	\end{vmatrix} =
	u'_xv'_y - v'_xu'_y
	$$
	По условию Коши-Римана, это равно:
	$$
	(u'_x)^2 + (v'_x)^2 = |f'(z_0)|^2 \neq 0\text{ — по условию.}
	$$
	По вещественной теореме об обратной функции существует окрестность точки $z_0$, в которой $f$ обратима и $f^{-1}$ в
	соответствующей окрестности точки $w_0$ непрерывна. Так как $f$ и $f^{-1}$ непрерывны, то
	$\Delta z \to 0 \iff \Delta w \to 0$.
	$$
	(f^{-1})'(w_0) = \lim_{\Delta w \to 0}\frac{\Delta f^{-1}}{\Delta w} = \lim_{\Delta w \to 0}
	\frac{\Delta z}{\Delta f} = \lim_{\Delta z \to 0}\frac{\Delta z}{\Delta f} =
	\frac{1}{\lim\limits_{\Delta z \to 0}\frac{\Delta f}{\Delta z}} = \frac{1}{f'(z_0)}
	$$
\end{proof}
\begin{examples}
	\begin{itemize}
		\item $f(z) = z^n$ — регулярна в $\complex$, $f'(z) = nz^{n-1}$;
		\item многочлены регулярны в $\complex$;
		\item $f(z) = \frac{1}{z}$ регулярна в $\complex\textbackslash \{0\}$, $f'(z) = -\frac{1}{z^2}$;
		\item дробно-линейная функция $\frac{az + b}{cz + d}$ регулярна в $\complex \textbackslash
		\{-\frac{d}{c}\}$.
	\end{itemize}
\end{examples}

\lparagraph{Геометрический смысл аргумента производной}

\begin{definition}
	Гладкая кривая в $\complex$ — это кривая, у которой существует параметризация $\gamma(t)$, являющаяся простым
	гладким путём: $\gamma'(t) \neq 0\;\forall t \in [a, b]$.
\end{definition}

Касательный вектор к кривой $\gamma$ в точке $z_0 = \gamma(t_0)$ есть $\gamma'(t_0)$, и он не зависит от
параметризации. Вспомним также, что угол между гладкими кривыми в точке их пересечения есть угол между их касательными
в этой точке.

Пусть функция $f$ голоморфна в области $G$, $z_0 = \gamma(t_0) \in G$, $\gamma: [a, b] \to \complex$ — гладкая
кривая, проходящая
через точку $z_0$, $\Gamma(t) = f(\gamma(t))$ — образ кривой $\gamma$. Касательный вектор к $\Gamma$ в точке
$w_0 = f(z_0)$ есть $\Gamma'(t_0)$.
$$
\Gamma'(t_0) = f'(z_0) \cdot \gamma'(t_0) \implies \Arg \Gamma'(t_0) = \arg f'(z_0) + \Arg \gamma'(t_0)
$$
И можно увидеть геометрический смысл аргумента производной: $\arg f'(z_0)$ — это угол, на который поворачивается
касательная к любой кривой в точке $z_0$ под действием $f$.

\lparagraph{Понятие конформности}

\begin{definition}
	Отображение $f$ называется \emph{конформным в точке} $z_0$, если оно сохраняет углы между кривыми в $z_0$
	(с учётом направления). $f$ называется \emph{конформным в области}~$G$, если оно конформно во всех точках области
	$G$ и однолистно (взаимно однозначно).
\end{definition}
\begin{examples}
\begin{itemize}
	\item если $f$ голоморфна в точке $z_0$ и $f'(z_0) \neq 0$, то $f$ конформно в $z_0$.
	\item $f(z) = z^2$. $f'(z) = 2z \implies f$ конформно в любой точке $z \neq 0$.

	Допустим, $z_0 = i$. $\arg f'(z_0) = \frac{\pi}{2}$.
	\begin{center}
	\begin{tikzpicture}[scale=2]
		\draw[axis] (-0.8, 0)  -- (0.8, 0) node(xline1)[right]{$x$};
	    \draw[axis] (0,-0.1) -- (0, 0.8) node(yline1)[above]{$y$};
	    \node at (0, 0.3) (i) {};
	    \fill[black] (i) circle (.03);
	    \node at ($(i) + (-0.1, 0)$) {$i$};

		\draw[axis] ($(-0.8, 0) + (2, 0)$)  -- ($(0.8, 0) + (2, 0)$) node(xline2)[right]{$x$};
	    \draw[axis] ($(0,-0.1) + (2, 0)$) -- ($(0, 0.8) + (2, 0)$) node(yline2)[above]{$y$};
	    \node at ($(-0.3, 0) + (2, 0)$) (imgi) {};
	    \fill[black] (imgi) circle (.03);
	    \node at ($(imgi) + (0, -0.15)$) {$-1$};

	    \node at ($(0.5, 0.5)$) (dummy1) {};
	    \node at ($(-0.5, 0.5) + (2, 0)$) (dummy2) {};
	    \path (dummy1) edge[pil, bend left=45] node[above] {$f$} (dummy2) {};
	\end{tikzpicture}
	\end{center}

	При $z_0 = 0$ конформности нет — углы между кривыми, проходящими через $0$, не сохраняются:
	\begin{center}
	\begin{tikzpicture}[scale=2,]
		\draw[axis] (-0.8, 0)  -- (0.8, 0) node(xline1)[right]{$x$};
	    \draw[axis] (0,-0.1) -- (0, 0.8) node(yline1)[above]{$y$};
	    \draw (0, 0) -- (0.4, 0.6) node(beta)[below, rotate=57]{\tiny $\arg z = \beta$};
	    \draw (0, 0) -- (0.6, 0.3) node(alpha)[below, rotate=27]{\tiny $\arg z = \alpha$};

		\draw[axis] ($(-0.8, 0) + (2, 0)$)  -- ($(0.8, 0) + (2, 0)$) node(xline2)[right]{$x$};
	    \draw[axis] ($(0,-0.1) + (2, 0)$) -- ($(0, 0.8) + (2, 0)$) node(yline2)[above]{$y$};
	    \draw ($(0, 0) + (2, 0)$) -- ($(-0.2, 0.48) + (2, 0)$) node(beta2)[below, rotate=-66]{\tiny $\arg w = 2\beta$};
	    \draw ($(0, 0) + (2, 0)$) -- ($(0.27, 0.36) + (2, 0)$) node(alpha2)[below, rotate=54]
	    	{\tiny $\arg w = 2\alpha$};

	    \node at ($(0.5, 0.5)$) (dummy1) {};
	    \node at ($(-0.5, 0.5) + (2, 0)$) (dummy2) {};
	    \path (dummy1) edge[pil, bend left=45] node[above] {$f$} (dummy2) {};
	\end{tikzpicture}
	\end{center}
	Видим, что угол межу прямыми составлял $\beta - \alpha$, а после действия функции $f$ стал $2(\beta - \alpha)$.
\end{itemize}
\end{examples}

\section{Элементарные функции комплексной переменной}

Изучим свойства некоторых важных функций.

\lparagraph{Целая степенная функция $f(z) = z^n$, $n \in \mb N$}

При $n = 1$ функция $f(z) = z$ регулярна на всей комплексной плоскости и конформна. При $n \geqslant 2$ $f$ регулярна
в $\complex$, а её производная $f'(z) = nz^{n-1}$.

$$
z_1^n = z_2^n \iff r_1^ne^{in\varphi_1} = r_2^ne^{in\varphi_2} \iff
\begin{cases}
	r_1 = r_2\\
	\varphi_1 - \varphi_2 = \frac{2\pi k}{n},\quad k \in \mb Z
\end{cases}
(\ast)
$$
Это означает, что $f$ взаимно однозначно в области $G$ тогда и только тогда, когда $G$ не содержит пар точек,
удовлетворяющих условию $(\ast)$. Пример такой области:
$$
G_k = \bigg\{\frac{2\pi k}{n} < \arg z < \frac{2\pi(k + 1)}{n}\bigg\}
$$
$f$ конформно отображает $G_k$ на $\complex \textbackslash \real_+$:

\begin{center}
	\begin{tikzpicture}[scale=4]
		\draw[axis] (-0.8, 0)  -- (0.8, 0) node(xline1)[right]{$x$};
	    \draw[axis] (0,-0.6) -- (0, 0.6) node(yline1)[above]{$y$};

	    \draw (0, 0) -- ++(45:0.6) node(line11)[right]{$\frac{2\pi}{n}$};
	    \draw (0, 0) -- ++(30:0.6);
	    \draw (0, 0) -- ++(15:0.6) node(line21)[right]{$\varphi$};
	    \draw[densely dashed] (0.2, 0) node(arc1)[below] {$r$} arc (0:45:0.2);
	    \draw[densely dashed] (0.4, 0) arc (0:45:0.4);

		  \draw[axis] ($(-0.8, 0) + (2, 0)$)  -- ($(0.8, 0) + (2, 0)$) node(xline2)[right]{$x$};
	    \draw[axis] ($(0,-0.6) + (2, 0)$) -- ($(0, 0.6) + (2, 0)$) node(yline2)[above]{$y$};

	    \draw ($(0, 0) + (2, 0)$) -- ++(120:0.6) node(line12)[right]{$n\varphi$};
	    \draw ($(0, 0) + (2, 0)$) -- ++(240:0.6);
	    \draw[densely dashed] ($(0.2, 0) + (2, 0)$) node(arc2)[below] {$r^n$} arc (0:360:0.2);
	    \draw[densely dashed] ($(0.4, 0) + (2, 0)$) arc (0:360:0.4);

	    \node at ($(0.5, 0.5)$) (dummy1) {};
	    \node at ($(-0.5, 0.5) + (2, 0)$) (dummy2) {};
	    \path (dummy1) edge[pil, bend left=45] node[above] {$f$} (dummy2) {};
	\end{tikzpicture}
\end{center}

Чтобы продолжить изучение элементарных функций, нам нужно сделать отступление и ввести понятие о непрерывных
ветвях и точках ветвления.

\lparagraph{Точки ветвления многозначной функции}

\begin{definition}
	Пусть $f$ — многозначная функция. Говорят, что в области $G$ выделена \emph{непрерывная ветвь} $f$, если любой
	точке из этой области сопоставлено одно значение $f(z)$ так, что полученная однозначная функция непрерывна.
\end{definition}

Аналогично определяется и непрерывная ветвь вдоль пути.

\begin{remark}
	Ни существование, ни единственность непрерывной ветви не гарантируются.
\end{remark}

\begin{definition}
	$z_0$ называется \emph{точкой ветвления} функции $f$, если в любой окрестности этой точки при обходе её по любому
	замкнутому пути любая непрерывная ветвь $f$ вдоль этого пути получает ненулевое приращение.
\end{definition}

Чтобы в $G$ существовала непрерывная ветвь $f$, необходимо, чтобы $G$ не содержала путей, обходящих точку ветвления.

\begin{example}
	$f(z) = \Arg z$. $0$ — единственная конечная точка ветвления функции $f$. Непрерывная ветвь $\Arg$ существует в
	области, если в этой области нельзя обойти точку $0$. Например, $\complex \textbackslash \real_+$.
\end{example}

\lparagraph{Функция $\sqrt[n]{z}$, $n \in \mb N$, $n \geqslant 2$}

Эта функция определена в области $\complex \textbackslash \{0\}$ и является $n$-значной.
$$
\sqrt[n]{z} = \sqrt[n]{|z|} \cdot e^{\frac{i\Arg z}{n}}
$$

Непрерывные ветви корня существуют там же, где и непрерывные ветви $\Arg$ — в областях, где нельзя обойти $0$. В каждой
такой области существует $n$ непрерывных ветвей:
$$
(\sqrt[n]{z})_k = \sqrt[n]{|z|} \cdot e^{\frac{i\arg z}{n} + \frac{2\pi k i}{n}},\quad k = 0,...,n-1
$$

К каждой ветви применима теорема об обратной функции:
$$
w = (\sqrt[n]{z})_k,\quad z = w^n
$$
$$
(\sqrt[n]{z})'_k = \frac{1}{z'} = \frac{1}{nw^{n - 1}} = \frac{1}{n}z^{\frac{1}{n} - 1}
$$

Отображение $f$ конформно в любой точке $z \neq 0$. В качестве области с непрерывной ветвью можно взять
$\complex \textbackslash \real_+$:

\begin{center}
	\begin{tikzpicture}[scale=4,]
		\draw[axis] (-0.8, 0)  -- (0.8, 0) node(xline1)[right]{$x$};
	    \draw[axis] (0,-0.6) -- (0, 0.6) node(yline1)[above]{$y$};

	    \draw[dotted, line width=3pt] (0, 0) -- (0.73, 0);
	    \draw (0, 0) -- ++(120:0.6) node(line11)[right]{$\varphi$};
	    \draw (0, 0) -- ++(240:0.6);
	    \draw[densely dashed] (0.2, 0) node(arc1)[below] {$r$} arc (0:360:0.2);
	    \draw[densely dashed] (0.4, 0) arc (0:360:0.4);

		\draw[axis] ($(-0.8, 0) + (2, 0)$)  -- ($(0.8, 0) + (2, 0)$) node(xline2)[right]{$x$};
	    \draw[axis] ($(0,-0.6) + (2, 0)$) -- ($(0, 0.6) + (2, 0)$) node(yline2)[above]{$y$};

	    \draw ($(0, 0) + (2, 0)$) -- ++(45:0.6) node(line12)[right]{};
	    \draw ($(0, 0) + (2, 0)$) -- ++(30:0.6);
	    \draw ($(0, 0) + (2, 0)$) -- ++(15:0.6) node(line21)[right]{$\frac{\varphi}{n}$};
	    \draw[densely dashed] ($(0.2, 0) + (2, 0)$) node(arc2)[below] {$\sqrt[n]{r}$} arc (0:45:0.2);
	    \draw[densely dashed] ($(0.4, 0) + (2, 0)$) arc (0:45:0.4);

	    \node at ($(0.5, 0.5)$) (dummy1) {};
	    \node at ($(-0.5, 0.5) + (2, 0)$) (dummy2) {};
	    \path (dummy1) edge[pil, bend left=45] node[above] {$(\sqrt[n]{z})_k$} (dummy2) {};
	\end{tikzpicture}
\end{center}

Далее главной ветвью корня будем называть главную ветвь $\Arg$.

\lparagraph{Экспоненциальная функция}

\begin{definition}
	Пусть $z = x + iy$. Тогда $e^z = e^x(\cos y + i\sin y)$.
\end{definition}

\begin{proposition}
	Свойства экспоненциальной функции:
\begin{enumerate}
	\item Комплексная экспоненциальная функция есть продолжение вещественной;
	\item\label{exponential_regularity} $f(z) = e^z$ регулярна в $\complex$, $f'(z) = e^z$;
	\item $e^z \neq 0\; \forall z \in \complex$;
	\item $f$ конформна в любой точке комплексной плоскости;
	\item\label{exponential_addmult} $e^{z_1+z_2} = e^{z_1}\cdot e^{z_2}$;
	\item Формула Эйлера: $e^{i\varphi} = \cos \varphi + i\sin \varphi$;
	\item Функция $f$ $2\pi i$-периодична: $e^{z + 2\pi i} = e^z$;
	\item $e^{z_1} = e^{z_2} \iff z_1 - z_2 = 2\pi ki$, $k \in \mb Z$. То есть $f$ взаимно
	однозначна в области $G$, если $G$ не содержит точек $z_1$, $z_2$ таких, что
	$z_1 - z_2 = 2\pi ki$;
	\item $|e^z| = e^x$, $\Arg e^z = y + 2\pi k$, $k \in \mb Z$.
\end{enumerate}
\end{proposition}
\begin{proof}
	Мы не будем доказывать все перечисленные свойства, так как большинство из них очевидны.
	Докажем свойства \ref{exponential_regularity} и \ref{exponential_addmult}.

	Для доказательства свойства \ref{exponential_regularity} нужно проверить дифференцируемость
	вещественной и мнимой части, а также условия Коши-Римана.
	Пусть $f = u + iv$. $u = e^x\cos y$, $v = e^x\sin y$ — эти функции дифференцируемы в $\real^2$.
	$u'_x = e^x\cos y = v'y$, $u'_y = -e^x\sin y = -v'_x$. Условия Коши-Римана выполнены.
	$$
	f'(z) = u'_x + iv'_x = e^x \cos y + ie^x\sin y = e^z
	$$

	Свойство \ref{exponential_addmult} доказывается простейшими преобразованиями:
	$$
	e^{z_1}\cdot e^{z_2} = e^{x_1}({\cos y_1 + i\sin y_1})\cdot e^{x_2}(\cos y_2 + i\sin y_2) =
	$$
	$$
	= e^{x_1}e^{x_2}(\cos y_1 \cos y_2 - \sin y_1 \sin y_2 + i(\cos y_1 \sin y_2 + \sin y_1 \cos y_2)) =
	$$
	$$
	= e^{x_1 + x_2}(\cos (y_1 + y_2) + i\sin (y_1 + y_2)) = e^{z_1 + z_2}
	$$
\end{proof}
\begin{example}
	$G = \{0 < \Img z < 2\pi\}$
	\begin{center}
	\begin{tikzpicture}[scale=3]
		\draw[axis] (-0.8, 0)  -- (0.8, 0) node(xline1)[right]{$x$};
	    \draw[axis] (0,-0.6) -- (0, 0.6) node(yline1)[above]{$y$};

	    \node[left] at (0, 0.4) {$2\pi$};
	    \draw (-0.8, 0.2) -- (0.8, 0.2);
	    \draw (-0.8, 0.06) -- (0.8, 0.06);
	    \draw[densely dashed] (0.1, 0) -- (0.1, 0.3);
	    \draw[densely dashed] (0.2, 0) -- (0.2, 0.3);
	    \path[fill=black,opacity=0.07] (-0.8, 0.3) -- (0.8, 0.3) -- (0.8, 0) -- (-0.8, 0) -- cycle;

		\draw[axis] ($(-0.8, 0) + (2, 0)$)  -- ($(0.8, 0) + (2, 0)$) node(xline2)[right]{$x$};
	    \draw[axis] ($(0,-0.6) + (2, 0)$) -- ($(0, 0.6) + (2, 0)$) node(yline2)[above]{$y$};

	    \draw[densely dashed] ($(0, 0) + (2, 0)$) circle (0.3);
	    \draw[densely dashed] ($(0, 0) + (2, 0)$) circle (0.2);
	    \draw ($(0, 0) + (2, 0)$) -- ++(40:0.6);
	    \draw ($(0, 0) + (2, 0)$) -- ++(240:0.6);
	    \draw[dotted, line width=3pt] ($(0, 0) + (2, 0)$) -- ($(0.73, 0) + (2, 0)$);
	    \path[fill=black,opacity=0.07] ($(-0.8, 0.6) + (2, 0)$) -- ($(0.8, 0.6) + (2, 0)$) --
	    	($(0.8, -0.6) + (2, 0)$) -- ($(-0.8, -0.6) + (2, 0)$) -- cycle;

	    \node at ($(0.5, 0.5)$) (dummy1) {};
	    \node at ($(-0.5, 0.5) + (2, 0)$) (dummy2) {};
	    \path (dummy1) edge[pil, bend left=45] node[above] {$e^z$} (dummy2) {};
	\end{tikzpicture}
	\end{center}
\end{example}

\lparagraph{Логарифмическая функция}

\begin{definition}
	Пусть $z \neq 0$. $w$ называется \emph{логарифмом} $z$, если $e^w = z$.
\end{definition}

Если $w = u + iv$, то $e^w = e^{u + iv} = z$ равносильно $e^u = |z|$, $v = \Arg z$, то есть $u = \ln |z|$.
Итого, для всех $z \neq 0$ существует бесконечно много логарифмов. Обозначим за $\Ln z$ множество всех логарифмов числа $z$:
$$
\Ln z = \{\ln |z| + i\arg z + 2\pi ki\, |\, k \in \mb Z\}
$$
Тогда $\ln z = \ln |z| + i\arg z$ будет \emph{главным значением логарифма}.

Итак, $\Ln$ — многозначная функция на $\complex \textbackslash \{0\}$ с точкой ветвления $0$.

\begin{proposition}
	Свойства логарифмической функции:
	\begin{enumerate}
		\item $\Ln (z_1\cdot z_2) = \Ln z_1 + \Ln z_2$;
		\item В области, в которой существует непрерывная ветвь логарифма, ветвей бесконечно много и они отличаются на
		$2\pi ki$. Каждая ветвь есть обратная к экспоненциальной функция.
	\end{enumerate}
\end{proposition}
\begin{proof}
	\begin{enumerate}
		\item Доказываем два включения. Пусть $w \in \Ln z_1 + \Ln z_2$. Тогда $w$ представимо в виде суммы $w_1 + w_2$, где
		$w_1 \in \Ln z_1$, $w_2 \in \Ln z_2$. Отсюда $e^w = e^{w_1}\cdot e^{w_2} = z_1z_2$, поэтому $w \in \Ln (z_1z_2)$.
		Обратно, пусть теперь $w \in \Ln (z_1z_2)$. Тогда $e^w = z_1z_2$. Возьмём $w_1 = \ln z_1$, $w_2 = w - w_1$.
		Получим, что $e^{w_1} = z_1$, $e^{w_2} = e^{w - w_1} = \frac{e^w}{e^{w_1}} = \frac{z_1z_2}{z_1} = z_2$, поэтому
		$w_2 \in \Ln z_2$. Таким образом, $w = w_1 + w_2$, $w_1 \in \Ln z_1$, $w_2 \in \Ln z_2$, отсюда
		$w \in \Ln z_1 + \Ln z_2$.
		\item Пусть $(\ln z)_k = \ln |z| + i\arg z + 2\pi ki$, $k \in \mb Z$.
		$$
		(\ln z)'_k = \frac{1}{(e^w)'} = \frac{1}{e^w} = \frac{1}{z}
		$$
	\end{enumerate}
\end{proof}

\lparagraph{Функция Жуковского}

\begin{definition}
	$f(z) = \frac{1}{2}(z + \frac{1}{z})$, где $z \in \complex \textbackslash \{0\}$, называется \emph{функцией Жуковского}.
\end{definition}

Функция Жуковского регулярна на области определения, $f'(z) = \frac{1}{2}(1 - \frac{1}{z^2})$. Она также конформна во всех 
точках кроме $0$ и $\pm 1$.
$$
f(z_1) = f(z_2) \iff z_1 + \frac{1}{z_1} = z_21 + \frac{1}{z_2} \iff (z_1 - z_2) + \frac{z_2 - z_1}{z_1z_2} = 0 \iff
$$
$$
\iff (z_1 - z_2)(1 - \frac{1}{z_1z_2}) = 0 \iff z_1 = z_2\text{ либо } z_1z_2 = 1.
$$

Таким образом, $f$ взаимно однозначно в любой области, не содержащей пар точек $z_1z_2 = 1$. Например,
$G = \{0 < |z| < 1\}$ или $G = \{|z| > 1\}$.

Теперь воспользуемся показательной формой комплексного числа и применим к ней функцию Жуковского:
$$
z = re^{i\varphi},\quad f(z) = \frac{1}{2}(re^{i\varphi} + \frac{1}{r}e^{-i\varphi}) = \frac{1}{2}(r + \frac{1}{r})\cos \varphi
+ i\frac{1}{2}(r - \frac{1}{r})\sin \varphi
$$

Видно, что окружность с центром в $0$ и радиусом $r_0$, где $0 < r_0 < 1$, переходит в эллипс, обходимый в
отрицательном направлении:
$$
w = a\cos \varphi + ib\sin \varphi, \quad a = \frac{1}{2}(r_0 + \frac{1}{r_0}), \quad b = -\frac{1}{2}(\frac{1}{r_0} - r_0)
$$

Найдём фокусы эллипса. Для эллипса, заданного уравнением $\frac{x^2}{a^2} + \frac{y^2}{b^2} = 1$ они равны
$\pm \sqrt{a^2 - b^2}$. Легко проверить, что фокусы равны $\pm 1$.

При $r_0 \to 0$ эллипс уходит на бесконечность, при $r_0 \to 1$ — <<схлопывается>> в отрезок $[-1, 1]$, проходимый дважды в 
противоположных направлениях. Значит, функция Жуковского конформно отображает $G = \{0 < |z| < 1\}$ на
$\complex \textbackslash [-1, 1]$.

Луч $\{\arg z = \varphi_0\}$, $0 < r < 1$ переходит в кривую:
$$
w = a\frac{1}{2}(r + \frac{1}{r}) -\frac{1}{2}ib(\frac{1}{r} - r), \quad a = \cos \varphi_0, \quad b = \sin \varphi_0
$$

Кривая, задаваемая уравнением $\frac{x^2}{a^2} - \frac{y^2}{b^2} = 1$ ($a \neq 0$, $b \neq 0$), — это гипербола, вернее,
в данном случае, половина
одной ветви гиперболы, лежащая в нужном квадранте. Фокусы гиперболы равны $\pm \sqrt{a^2 + b^2} = \pm 1$

Рассмотрим случаи, когда $a = 0$ или $b = 0$:
\begin{itemize}
	\item при $\varphi_0 = 0$ образом будет луч $(1, \infty)$;
	\item при $\varphi_0 = \frac{pi}{2}$ образ — луч $(0, -i\infty)$;
	\item при $\varphi_0 = pi$ — луч $(-\infty, -1)$;
	\item при $\varphi_0 = \frac{3pi}{2}$ — луч $(0, i\infty)$.
\end{itemize}

\begin{center}
	\begin{tikzpicture}[scale=4]
		\draw[axis] (-0.8, 0)  -- (0.8, 0) node(xline1)[right]{$x$};
	    \draw[axis] (0,-0.6) -- (0, 0.6) node(yline1)[above]{$y$};

		\draw[dotted] (0, 0) circle (0.4);
		\draw[densely dashed] (0, 0) circle (0.2);
		\draw (0, 0) -- ++(30:0.7);
		\node[below] at (0.2, 0) {$r_0$};
		\node[below] at (0.4, 0) {$1$};
		\path[fill=black,opacity=0.07,even odd rule] circle (0.4) circle (0.2); 


    	\begin{scope}[shift={(2,0)}]
    		\draw[axis] (-0.8, 0)  -- (0.8, 0) node(xline2)[right]{$x$};
	    	\draw[axis] (0,-0.6) -- (0, 0.6) node(yline2)[above]{$y$};
    		\draw[densely dashed] (0, 0) circle [x radius = 0.5, y radius=0.3];
    		\path[fill=black,opacity=0.07] circle [x radius = 0.5, y radius=0.3]; 
			\draw plot[domain=0:2] ({0.2*cosh(\x)},{-abs(0.15*sinh(\x))});
			\node at (-0.35, 0) (minusone) {};
			\node at (0.35, 0) (one) {};
			\node at ($(minusone) + (0, -0.08)$) {$-1$};
			\node at ($(one) + (0, -0.08)$) {$1$};
	    	\fill[black] (minusone) circle (.02);
	    	\fill[black] (one) circle (.02);
		\end{scope}

	    \node at ($(0.5, 0.5)$) (dummy1) {};
	    \node at ($(-0.5, 0.5) + (2, 0)$) (dummy2) {};
	    \path (dummy1) edge[pil, bend left=45] node[above] {$f$} (dummy2) {};
	\end{tikzpicture}
\end{center}

\lparagraph{Функция, обратная к функции Жуковского}

Исследуем функцию $z = w + \sqrt{(w - 1)(w + 1)}$. Это двузначная функция.
$$
\Arg \sqrt{(w - 1)(w + 1)} = \frac{1}{2}(\Arg (w - 1) + \Arg (w + 1))
$$
$$
\Delta_\gamma\arg \sqrt{(w - 1)(w + 1)} = \frac{1}{2}(\Delta_\gamma\arg (w - 1) + \Delta_\gamma\arg (w + 1))
$$
Здесь $\Delta_\gamma\arg$ означает приращение аргумента при обходе вдоль кривой $\gamma$. Узнаем, где существует непрерывная 
ветвь функции, обратной к функции Жуковского:

\begin{itemize}
	\item если $\gamma$ не обходит ни $-1$, ни $1$, то $\Delta_\gamma\arg \sqrt{(w - 1)(w + 1)} = 0$ - хорошо;
	\item если $\gamma$ обходит $1$, но не обходит $-1$, то $\Delta_\gamma\arg \sqrt{(w - 1)(w + 1)} = \pi$ - плохо;
	\item если $\gamma$ обходит $-1$, но не обходит $1$, то $\Delta_\gamma\arg \sqrt{(w - 1)(w + 1)} = \pi$ - плохо;
	\item если $\gamma$ обходит и $1$, и $-1$, то $\Delta_\gamma\arg \sqrt{(w - 1)(w + 1)} = 2\pi$ - хорошо.
\end{itemize}

Таким образом, непрерывная ветвь существует в областях, в которых нельзя обойти ровно одну точки из $\pm 1$. Например,
$\complex \textbackslash [-1, 1]$.

\lparagraph{Тригонометрические функции}

\begin{definition}
	$\sin z = \frac{e^{iz} - e^{-iz}}{2i}$, $\cos z = \frac{e^{iz} + e^{-iz}}{2}$, $\tg z = \frac{\sin z}{\cos z}$,
	$\ctg z = \frac{\cos z}{\sin z}$.
\end{definition}

\end{document}
